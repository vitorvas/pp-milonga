\documentclass{anstrans}

%%%% packages and definitions (optional)
\usepackage{graphicx} % allows inclusion of graphics
\usepackage{booktabs} % nice rules (thick lines) for tables
\usepackage{microtype} % improves typography for PDF


\newcommand{\SN}{S$_N$}
\renewcommand{\vec}[1]{\bm{#1}} % vector is bold italic
\newcommand{\vd}{\bm{\cdot}} % slightly bold vector dot
\newcommand{\grad}{\vec{\nabla}} % gradient
\newcommand{\ud}{\mathop{}\!\mathrm{d}} % upright derivative symbol


%%%% changes from the original 'anstrans' class
\usepackage{fancyhdr} % allows headers and footers

\renewcommand\headrule{} % remove underline in the header
\setcounter{secnumdepth}{1}
\renewcommand{\thesection}{\Roman{section}.}
\renewcommand{\thesubsection}{\arabic{subsection}.}
\renewcommand{\thesubsubsection}{\Alph{subsubsection}.}
\makeatletter
\renewcommand*{\@seccntformat}[1]{\csname the#1\endcsname\hspace{1mm}}
\makeatother


%%%% Header
%\pagestyle{fancy}
%\fancyhf{}
%\fancyhead[L]{\fontsize{9}{9} \itshape
%M\&C 2017 - International Conference on Mathematics \& Computational Methods Applied to Nuclear Science \& Engineering,
%\\ Jeju, Korea, April 16-20, 2017, on USB (2017)
%}


%%%% Maketitle
\title{Parallelized milonga}
\author{Vitor V. A. Silva,$^{*}$ and Germ\'an Theler$^{\dagger}$}

\institute{
$^{*}$Centro de Desenvolvimento da Tecnologia Nuclear, CEP: 30......
Belo Horizonte - MG, Brazil
\and
$^{\dagger}$Seamples, Rafaela, Argentina
}

\email{vitors@cdtn.br \and theler@seamplex.com}

% Optional disclaimer: remove this command to hide
\disclaimer{Notice: this manuscript is a work of fiction. Any resemblance to
actual articles, living or dead, is purely coincidental.}


%%%% Abstract
\begin{document}
\vspace*{-42pt}
\begin{strip}
\centering{\parbox{153mm}{{\bf Abstract} \itshape - 
This is the abstract one day I'll write
}\par}
\vspace*{14pt}
\end{strip}


%%%%%%%%%%%%%%%%%%%%%%%%%%%%%%%%%%%%%%%%%%%%%%%%%%%%%%%%%%%%%%%%%%%%%%%%%%%%%%%%
\section{Introduction}

Para falar do milonga em paralelo, é preciso antes contextualizar o milonga. Para
isso, é obrigatório falar do wasora. É no wasora que é feito o controle de malhas.
Sendo assim, em duas abordagens paralelas diferentes:
\begin{itemize}
\item Paralelizar apenas a solução das matrizes via SLEPc e PETSc;
  \item Paralelizar decompondo a malha completamente.
\end{itemize}

milonga makes extensive use of PETSc \cite{petsc} and SLEPc \cite{slepc} libraries and
its impossible to propose any paralelization scheme for milonga before devoting some
time to understand the way both of these libraries work.

%%%%%%%%%%%%%%%%%%%%%%%%%%%%%%%%%%%%%%%%%%%%%%%%%%%%%%%%%%%%%%%%%%%%%%%%%%%%%%%%
\section{Theory}

Equations look exceedingly pretty. Here is a 3-D, monoenergetic, steady-state
transport equation with isotropic scattering and an isotropic extraneous source:
\begin{subequations} \label{eqs:fullTransport}
\begin{multline} \label{eq:fullTransportVol}
  \vec{\Omega}\vd \grad \psi(\vec{x}, \vec{\Omega})
  + \sigma(\vec{x}) \psi (\vec{x}, \vec{\Omega})
\\ =
  \frac{\sigma_s(\vec{x})}{4\pi} \int_{4\pi} \psi(\vec{x},\vec{\Omega}')
  \ud\Omega' + \frac{q(\vec{x})}{4\pi}
  \equiv \frac{1}{4\pi} Q(\vec{x}) \,,
\end{multline}
inside $\vec{x} \in V$, $\vec{\Omega} \in 4\pi$, with an incident boundary
condition
\begin{equation} \label{eq:fullTransportBndy}
  \psi(\vec{x}, \vec{\Omega}) = \psi^b(\vec{x}, \vec{\Omega}) \,,
 \quad \vec{x} \in \partial V, \ \vec{\Omega} \vd \vec{n} < 0\,.
\end{equation}
\end{subequations}

%%%%%%%%%%%%%%%%%%%%%%%%%%%%%%%%%%%%%%%%%%%%%%%%%%%%%%%%%%%%%%%%%%%%%%%%%%%%%%%%
\section{Results and Analysis}
The results were interesting, so interesting in fact that we have decided to
present them here.

%%%%%%%%%%%%%%%%%%%%%%%%%%%%%%%%%%%%%%%%%%%%%%%%%%%%%%%%%%%%%%%%%%%%%%%%%%%%%%%%
\subsection{Subsection Goes Here}
The user must manually capitalize initial letters of a subsection heading.

For those who like equations in their papers, \LaTeX\ is a good choice. Here is
an equation for the Marshak diffusion boundary condition:
\begin{equation} \label{eq:marshak}
  4 J^- = \phi + 2 D \vec{n} \vd \grad \phi \,.
\end{equation}
If we so choose, we can effortlessly reference the equation later.

Another paragraph starts with Eq.~\eqref{eq:marshak} and sets $J^-$ to zero, a
vacuum boundary condition:
\begin{equation*}
  0 = \phi + \frac{2}{3} \frac{1}{\sigma} \vec{n} \vd \grad \phi \,.
\end{equation*}
The extrapolation distance is $2/3$. A more detailed asymptotic analysis yields
an extrapolation distance of about $0.71045$.

Figure~\ref{fig:voltage} shows how a plot might conceivably look in your
document. Always place figures after they are referenced so as not to throw
off the reader. You can use symbols and different line styles to help
differentiate your results, especially if they are printed in black and white.
Note how Fig.~\ref{fig:voltage} uses dashed lines \verb|--| for the exact
solution, solid lines \verb|-| for the new method's solutions, and dotted lines
\verb|:| for existing inaccurate methods.
\begin{figure}[ht] % replace 't' with 'b' to force it to be on the bottom
  \centering
  \includegraphics{example_figure}
  \caption{Captions are flush with the left.}
  \label{fig:voltage}
\end{figure}

Later on, we can include a table, even one that spans two columns such as
Table~\ref{tab:widetable}.
%%%%%%%%%%%%%%%%%%%%%%%%%%%%%%%%%%%%%%%%
\begin{table*}[htb]
  \centering
\begin{tabular}{llllllllll}\toprule
      & $\phi_T(0)$      & $\phi_T(10)$      & $\phi_T(20)$      &
      $\phi_D(0)$      & $\phi_D(10)$      & $\phi_D(20)$      & $\rho$      &
      $\varepsilon$      & $N_\text{it}$
\\ \midrule
$c=0.999$  & 0.9038 & 20.63 & 31.24 & 0.9087 & 20.63 & 31.23 & 0.2192 & $10^{-7}$ & 15
\\
$c=0.990$  & 0.3675 & 13.04 & 24.7 & 0.3696 & 13.04 & 24.69 & 0.2184 & $10^{-7}$ & 15
\\
$c=0.900$  & 0.009909 & 4.776 & 17.64 & 0.009984 & 4.786 & 17.63 & 0.2118 & $10^{-7}$ & 14
\\
$c=0.500$  & $6.069\times 10^{-5}$ & 2.212 & 15.53 & 6.213$\times 10^{-5}$ & 2.239 & 15.53 & 0.2068 & $10^{-7}$ & 13
\\
\bottomrule
\end{tabular}
  \caption{This is an example of a really wide table which might not normally
  fit in the document.}
  \label{tab:widetable}
\end{table*}
%%%%%%%%%%%%%%%%%%%%%%%%%%%%%%%%%%%%%%%%
Notice how the table reference uses a Roman numeral
for its numbering scheme, whereas the figure reference uses an Arabic numeral.
For one-column tables, use the \verb|table| environment; two-column tables use
\verb|table*|. The same applies to figures.

%%%%%%%%%%%%%%%%%%%%%%%%%%%%%%%%%%%%%%%%%%%%%%%%%%%%%%%%%%%%%%%%%%%%%%%%%%%%%%%%
\subsection{Another Subsection}
Excessive sectioning in a three-page document is discouraged, but here are more
subsections to demonstrate compliance with the ANS formatting guidelines.

\subsubsection{Third-level Heading}
This subsubsection shows compliance with the ANS-specified standard. This level
of heading should be used rarely.

\subsubsection{Another Such Heading}
And, if you really think you need a third-level heading, you should make sure
that your subsection needs at least two of them.

%%%%%%%%%%%%%%%%%%%%%%%%%%%%%%%%%%%%%%%%%%%%%%%%%%%%%%%%%%%%%%%%%%%%%%%%%%%%%%%%
\section{Conclusions}

%%%%%%%%%%%%%%%%%%%%%%%%%%%%%%%%%%%%%%%%%%%%%%%%%%%%%%%%%%%%%%%%%%%%%%%%%%%%%%%%
\appendix
\section{Appendix}

Numbering in the appendix is different:
\begin{equation} \label{eq:appendix}
  2 + 2 = 5\,.
\end{equation}
and another equation:
\begin{equation} \label{eq:appendix2}
  a + b = c\,.
\end{equation}

%%%%%%%%%%%%%%%%%%%%%%%%%%%%%%%%%%%%%%%%%%%%%%%%%%%%%%%%%%%%%%%%%%%%%%%%%%%%%%%%
\section{Acknowledgments}
This material is based upon work supported a Department of Energy Nuclear
Energy University Programs Graduate Fellowship.

%%%%%%%%%%%%%%%%%%%%%%%%%%%%%%%%%%%%%%%%%%%%%%%%%%%%%%%%%%%%%%%%%%%%%%%%%%%%%%%%
\bibliographystyle{ans} % Don't forget to run BibTeX !
\bibliography{milongap}

\end{document}
