\documentclass{anstrans}

%%%% packages and definitions (optional)
\usepackage{graphicx} % allows inclusion of graphics
\usepackage{booktabs} % nice rules (thick lines) for tables
\usepackage{microtype} % improves typography for PDF
\usepackage{verbatim} 

\newcommand{\SN}{S$_N$}
\renewcommand{\vec}[1]{\bm{#1}} % vector is bold italic
\newcommand{\vd}{\bm{\cdot}} % slightly bold vector dot
\newcommand{\grad}{\vec{\nabla}} % gradient
\newcommand{\ud}{\mathop{}\!\mathrm{d}} % upright derivative symbol


%%%% changes from the original 'anstrans' class
\usepackage{fancyhdr} % allows headers and footers

\renewcommand\headrule{} % remove underline in the header
\setcounter{secnumdepth}{1}
\renewcommand{\thesection}{\Roman{section}.}
\renewcommand{\thesubsection}{\arabic{subsection}.}
\renewcommand{\thesubsubsection}{\Alph{subsubsection}.}
\makeatletter
\renewcommand*{\@seccntformat}[1]{\csname the#1\endcsname\hspace{1mm}}
\makeatother


%%%% Header
%\pagestyle{fancy}
%\fancyhf{}
%\fancyhead[L]{\fontsize{9}{9} \itshape
%M\&C 2017 - International Conference on Mathematics \& Computational Methods Applied to Nuclear Science \& Engineering,
%\\ Jeju, Korea, April 16-20, 2017, on USB (2017)
%}


%%%% Maketitle
\title{An analysis of \textit{milonga} nuclear code aimed to multi-thread parallelization}
%\title{Milonga Nuclear Code implementation analisys aimed to future parallelization}
\author{Vitor Vasconcelos$^{*}$, Andr\'e A. C. dos Santos$^{*}$, Daniel Campolina$^{*}$ and Germ\'an Theler$^{\dagger}$}

\institute{
$^{*}$Centro de Desenvolvimento da Tecnologia Nuclear, CEP 31270-901,
Belo Horizonte - MG, Brazil
\and
$^{\dagger}$Seamples, Rafaela, Argentina
}

\email{\{vitors, aacs, campolina\}@cdtn.br \and theler@seamplex.com}

% Optional disclaimer: remove this command to hide
\disclaimer{Notice: this manuscript is a work of fiction. Any resemblance to
actual articles, living or dead, is purely coincidental.}


%%%% Abstract
\begin{document}
\vspace*{-42pt}
\begin{strip}
\centering{\parbox{153mm}{{\bf Abstract} \itshape - 
This is the abstract one day I'll write
}\par}
\vspace*{14pt}
\end{strip}


%%%%%%%%%%%%%%%%%%%%%%%%%%%%%%%%%%%%%%%%%%%%%%%%%%%%%%%%%%%%%%%%%%%%%%%%%%%%%%%%
\section{Introduction}

Before any atempt to describe \textit{milonga} nuclear code in parallel, it is
necessary to describe its context and how it works. In order to do that, is mandatory
to give a brief introduction of \textit{wasora}.

Wasora is... \cite{wasora}.

There are two first approaches to parallelize milonga:
\begin{itemize}
\item Parallelize only the solution of the matrices obtained by the discretization
  provided by either finite volumes of finite elements discretization schemes using
  SLEPc and PETSc libraries;
\item Parallelize using domain decomposition techniques.
\end{itemize}

milonga makes extensive use of PETSc \cite{petsc} and SLEPc \cite{Hernandez2005} libraries and
its impossible to propose any paralelization scheme for milonga before devoting some
time to understand the way both of these libraries work.

%%%%%%%%%%%%%%%%%%%%%%%%%%%%%%%%%%%%%%%%%%%%%%%%%%%%%%%%%%%%%%%%%%%%%%%%%%%%%%%%
\section{Theory}

\section{Profiling}
Before any attempt to parallelize any piece of software, a basic task must be performed:
software execution profiling. Profiling consists in a dynamic program analysis
that measures memory use, time complexity, number of function calls and many other characteristics of
a program.

The standard profiling tool for Linux systems is gprof [cite]. Gprof was used in a standard milonga
execution, with a mesh of about 300 thousand elements and fixed cross-sections in order to check
the any bottlenecks of its execution.

For the selected mesh and finite volumes problem, two functions were clearly identified as bottlenecks:
\begin{verbatim}
kd_nearest_i
hyperrect_dist_sq
\end{verbatim}

These functions belong to the third party code used to manipulate meshes. This chunk of code implements
a \textit{k-d tree} structure and its associated algorithms. This data structure, proposed by \cite{Bentley1975}
is aimed to partitioning a space and find neighobours of an element. Despite being the most used data
structure for the proposed objective, it performes poorly for the needs of \textit{milonga}.

The reason for this degraded performance is the ammount of data access necessary for finite volumes
and finite elements discretization schemes. One solution for this bottleneck is to change the data structure
in order to avoid the prohibitive number of neighbours searches. This can be accomplished storing neighbour
information at each node or element. Of course, the drawback of this solution is the increase in the memory
comsumption to store the data.

\subsection{Examples: 3dshape cube}

Some preliminary tests regarding a tethraedrical cube lead to the \texttt{mesh/tet4.c} file.
Specifically, in function \texttt{mesh\_four\_node\_tetrahedron\_dhdr}, there is a long case
statement which was replaced by a two if clauses function body, simply returning $-1$, $1$ or $0$
depending on function arguments. These tests lead to an astonishing slow run of milonga (alcor-lx).

\subsection{Scalability}

In order to assess how the performace of milonga degrades with the increase of the problem size, a \textit{benchmark}
test is proposed. This reference case is a modified version of the 2D PWR case provided with milonga \cite{milonga}
and consists of the same problem solved in three different discretizations, respectively increased. In other words,
the reference mesh was refined one time and the refined mesh was again refined. This process was carried by simply
using the \texttt{-refine} parameter of gmsh.

The mesh sizes are presented in table ~\ref{tab:meshes}.

\begin{table}[]
\centering
\caption{Comparative results for three meshes,}
\label{tab:meshes}
\begin{tabular}{crcr}
                            & \multicolumn{1}{c}{elements} & matrix size       & \multicolumn{1}{c}{\begin{tabular}[c]{@{}c@{}}time\\ elapsed {[}s{]}\end{tabular}} \\ \hline
\multicolumn{1}{c|}{Mesh 0} & 12,884                       & 25,768 x 25,768   & 11.47                                                                              \\ \hline
\multicolumn{1}{c|}{Mesh 1} & 51,536                       & 103,072 x 103,072 & 87.68                                                                              \\ \hline
\multicolumn{1}{c|}{Mesh 2} & 206,154                      & 412,288 x 412,288 & 773.33                                                                             \\ \hline
\end{tabular}
\end{table}


\subsection{Matrices aimed to parallel solution [This has no effect in improving milonga performance]}
A simple approach to make milonga take advantage of parallel architectures is to parallelize the
solution of the linear system generated by the finite volumes of finite elements methods.

The computations are carried in two steps:
\begin{enumerate}
\item Matrix building;
  \item Matrix solution.
\end{enumerate}

milonga uses the PETSc library to perform both tasks. In order to be able to solve the linear problem
in parallel, a special care must be taken in the proccess of building the matrices.


\subsection{Parallel Programming Patterns}
(James Reinders - Structure Parallel Programming presentation)
Parallel programming patterns where used to get communication between proccess after
domain decomposition but also to estimate the best decompositions considering the system
architecture. The partitions algorithm is parametrized, which allows fine tuning of performance
simply making parameters twinkling thus changinhg the memory ocuppancy/inter proccess communication ratio.



%Equations look exceedingly pretty. Here is a 3-D, monoenergetic, steady-state
%transport equation with isotropic scattering and an isotropic extraneous source:
%\begin{subequations} \label{eqs:fullTransport}
%\begin{multline} \label{eq:fullTransportVol}
%  \vec{\Omega}\vd \grad \psi(\vec{x}, \vec{\Omega})
%  + \sigma(\vec{x}) \psi (\vec{x}, \vec{\Omega})
%\\ =
%  \frac{\sigma_s(\vec{x})}{4\pi} \int_{4\pi} \psi(\vec{x},\vec{\Omega}')
%  \ud\Omega' + \frac{q(\vec{x})}{4\pi}
%  \equiv \frac{1}{4\pi} Q(\vec{x}) \,,
%\end{multline}
%inside $\vec{x} \in V$, $\vec{\Omega} \in 4\pi$, with an incident boundary
%condition
%\begin{equation} \label{eq:fullTransportBndy}
%  \psi(\vec{x}, \vec{\Omega}) = \psi^b(\vec{x}, \vec{\Omega}) \,,
% \quad \vec{x} \in \partial V, \ \vec{\Omega} \vd \vec{n} < 0\,.
%\end{equation}
%\end{subequations}

%%%%%%%%%%%%%%%%%%%%%%%%%%%%%%%%%%%%%%%%%%%%%%%%%%%%%%%%%%%%%%%%%%%%%%%%%%%%%%%%
\section{Results and Analysis}

Milonga runtime, profiling and memory demand before changes.
Milonga runtime, profiling and memory demand after changes

Assesment of changes.

%%%%%%%%%%%%%%%%%%%%%%%%%%%%%%%%%%%%%%%%%%%%%%%%%%%%%%%%%%%%%%%%%%%%%%%%%%%%%%%%
\subsection{Subsection Goes Here}
The user must manually capitalize initial letters of a subsection heading.

For those who like equations in their papers, \LaTeX\ is a good choice. Here is
an equation for the Marshak diffusion boundary condition:
\begin{equation} \label{eq:marshak}
  4 J^- = \phi + 2 D \vec{n} \vd \grad \phi \,.
\end{equation}
If we so choose, we can effortlessly reference the equation later.

Figure~\ref{fig:voltage} shows how a plot might conceivably look in your
document. Always place figures after they are referenced so as not to throw
off the reader. You can use symbols and different line styles to help
differentiate your results, especially if they are printed in black and white.
Note how Fig.~\ref{fig:voltage} uses dashed lines \verb|--| for the exact
solution, solid lines \verb|-| for the new method's solutions, and dotted lines
\verb|:| for existing inaccurate methods.
\begin{figure}[ht] % replace 't' with 'b' to force it to be on the bottom
  \centering
  \includegraphics{example_figure}
  \caption{Captions are flush with the left.}
  \label{fig:voltage}
\end{figure}

Later on, we can include a table, even one that spans two columns such as
Table~\ref{tab:widetable}.
%%%%%%%%%%%%%%%%%%%%%%%%%%%%%%%%%%%%%%%%
\begin{table*}[htb]
  \centering
\begin{tabular}{llllllllll}\toprule
      & $\phi_T(0)$      & $\phi_T(10)$      & $\phi_T(20)$      &
      $\phi_D(0)$      & $\phi_D(10)$      & $\phi_D(20)$      & $\rho$      &
      $\varepsilon$      & $N_\text{it}$
\\ \midrule
$c=0.999$  & 0.9038 & 20.63 & 31.24 & 0.9087 & 20.63 & 31.23 & 0.2192 & $10^{-7}$ & 15
\\
$c=0.990$  & 0.3675 & 13.04 & 24.7 & 0.3696 & 13.04 & 24.69 & 0.2184 & $10^{-7}$ & 15
\\
$c=0.900$  & 0.009909 & 4.776 & 17.64 & 0.009984 & 4.786 & 17.63 & 0.2118 & $10^{-7}$ & 14
\\
$c=0.500$  & $6.069\times 10^{-5}$ & 2.212 & 15.53 & 6.213$\times 10^{-5}$ & 2.239 & 15.53 & 0.2068 & $10^{-7}$ & 13
\\
\bottomrule
\end{tabular}
  \caption{This is an example of a really wide table which might not normally
  fit in the document.}
  \label{tab:widetable}
\end{table*}
%%%%%%%%%%%%%%%%%%%%%%%%%%%%%%%%%%%%%%%%
Notice how the table reference uses a Roman numeral
for its numbering scheme, whereas the figure reference uses an Arabic numeral.
For one-column tables, use the \verb|table| environment; two-column tables use
\verb|table*|. The same applies to figures.

%%%%%%%%%%%%%%%%%%%%%%%%%%%%%%%%%%%%%%%%%%%%%%%%%%%%%%%%%%%%%%%%%%%%%%%%%%%%%%%%
\subsection{Another Subsection}
Excessive sectioning in a three-page document is discouraged, but here are more
subsections to demonstrate compliance with the ANS formatting guidelines.

\subsubsection{Third-level Heading}
This subsubsection shows compliance with the ANS-specified standard. This level
of heading should be used rarely.

\subsubsection{Another Such Heading}
And, if you really think you need a third-level heading, you should make sure
that your subsection needs at least two of them.

%%%%%%%%%%%%%%%%%%%%%%%%%%%%%%%%%%%%%%%%%%%%%%%%%%%%%%%%%%%%%%%%%%%%%%%%%%%%%%%%
\section{Conclusions}

%%%%%%%%%%%%%%%%%%%%%%%%%%%%%%%%%%%%%%%%%%%%%%%%%%%%%%%%%%%%%%%%%%%%%%%%%%%%%%%%
\appendix
\section{Appendix}

Numbering in the appendix is different:
\begin{equation} \label{eq:appendix}
  2 + 2 = 5\,.
\end{equation}
and another equation:
\begin{equation} \label{eq:appendix2}
  a + b = c\,.
\end{equation}

%%%%%%%%%%%%%%%%%%%%%%%%%%%%%%%%%%%%%%%%%%%%%%%%%%%%%%%%%%%%%%%%%%%%%%%%%%%%%%%%
\section{Acknowledgments}
This material is based upon work supported a Department of Energy Nuclear
Energy University Programs Graduate Fellowship.

%%%%%%%%%%%%%%%%%%%%%%%%%%%%%%%%%%%%%%%%%%%%%%%%%%%%%%%%%%%%%%%%%%%%%%%%%%%%%%%%
\bibliographystyle{ans} % Don't forget to run BibTeX !
\bibliography{milongap}

\end{document}
